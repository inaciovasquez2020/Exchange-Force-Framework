\documentclass[11pt]{article}

\usepackage[T1]{fontenc}
\usepackage[utf8]{inputenc}
\usepackage{amsmath,amssymb,amsthm}
\usepackage{fullpage}
\usepackage{microtype}
\usepackage{enumitem}

\newtheorem{definition}{Definition}
\newtheorem{assumption}{Assumption}
\newtheorem{theorem}{Theorem}
\newtheorem{lemma}{Lemma}
\newtheorem{remark}{Remark}

\title{Exchange Force Framework (Phase I): A Structural Non--Fundamentality Result}
\author{Inacio F. Vasquez}
\date{January 3, 2026}

\begin{document}
\maketitle

\begin{center}
\textbf{Confidential --- NDA Required}\\
\textbf{Operational Rigidity Portfolio}\\
\textbf{Asset C4 --- Application Layer}
\end{center}

\begin{abstract}
This document defines Asset~C4 of the Operational Rigidity portfolio: the Exchange Force Framework (Phase~I). The framework establishes that so--called exchange forces are not fundamental interactions but emergent effects arising from symmetry constraints and mediator exchange within finite--capacity operational regimes. A unified effective--Hamiltonian description is provided, symmetry--driven and mediator--driven exchange are separated, and the Exchange Correlation Capacity (ECC) invariant is introduced as the structural control parameter. This asset instantiates the Unified Rigidity Framework (Asset~A) via the Operational Bridge Layer (Asset~B1).
\end{abstract}

\section{Position in the IP Stack}

\begin{definition}[Asset Role]
Asset~C4 applies the Unified Rigidity Framework to exchange phenomena across quantum mechanics and quantum field theory.
\end{definition}

\begin{remark}
This is a structural classification, not a force--model proposal.
\end{remark}

\section{Exchange Interactions: Operational Definition}

\begin{definition}[Exchange Interaction]
An exchange interaction is an effective interaction term arising from either (i) symmetry constraints on admissible state spaces or (ii) virtual mediator exchange, without introducing a new fundamental coupling.
\end{definition}

\begin{definition}[Operational Setting]
Systems are probed and evolved through admissible operations with bounded information and energy capacity.
\end{definition}

\section{Classification of Exchange Forces}

\begin{definition}[Type~I: Symmetry--Driven Exchange]
Type~I exchange arises purely from (anti)symmetrization of multi--particle state spaces under indistinguishability.
\end{definition}

\begin{definition}[Type~II: Mediator--Driven Exchange]
Type~II exchange arises from integrating out mediator fields, yielding effective interaction terms in a reduced description.
\end{definition}

\begin{remark}
Both types are effective, not fundamental.
\end{remark}

\section{Unified Effective Hamiltonian}

\begin{definition}[Effective Hamiltonian Form]
All exchange interactions admit a representation
\[
H_{\mathrm{eff}} = H_0 + H_{\mathrm{sym}} + H_{\mathrm{med}},
\]
where $H_{\mathrm{sym}}$ encodes symmetry--driven exchange and $H_{\mathrm{med}}$ encodes mediator--driven exchange.
\end{definition}

\begin{remark}
No independent \emph{exchange force} term is required.
\end{remark}

\section{Exchange Correlation Capacity (ECC)}

\begin{definition}[Exchange Correlation Capacity]
The Exchange Correlation Capacity (ECC) quantifies the maximum correlation strength sustainable by admissible exchange mechanisms under bounded operational capacity.
\end{definition}

\begin{assumption}[Finite ECC]
For any admissible system, ECC is finite and determined by symmetry class, mediator content, and operational resolution.
\end{assumption}

\section{Non--Fundamentality Theorem}

\begin{theorem}[Non--Fundamentality of Exchange Forces]
No exchange interaction constitutes a fundamental force. All exchange effects are derivable from symmetry constraints or mediator exchange within the unified effective--Hamiltonian framework.
\end{theorem}

\begin{remark}
This statement holds independently of interaction strength.
\end{remark}

\section{Rigidity via ECC}

\begin{theorem}[Exchange Rigidity (Phase~I)]
If ECC exceeds a system--specific threshold, macroscopic phase rigidity emerges. Conversely, unbounded exchange amplification is impossible under finite operational capacity.
\end{theorem}

\begin{remark}
This rigidity statement is structural and non--perturbative.
\end{remark}

\section{Defect Interpretation}

\begin{definition}[Exchange Defect]
An exchange defect is a hypothetical interaction behaving as a new fundamental force or exhibiting unbounded correlation amplification.
\end{definition}

\begin{lemma}[Defect Inadmissibility]
Any exchange defect violates finite operational capacity or symmetry constraints and is therefore forbidden by the Unified Rigidity Framework.
\end{lemma}

\section{Validation Checks}

\begin{itemize}
\item \textbf{Backtrack:} Removing symmetry or mediators eliminates exchange effects.
\item \textbf{Creeper:} Consistency with Heisenberg dimer and Yukawa bound states.
\item \textbf{Limit:} Mediator mass $\to \infty$ suppresses Type~II exchange.
\end{itemize}

\section{Scope and Phase Boundary}

\begin{remark}
Phase~I establishes classification and non--fundamentality. Phase~II addresses rigorous ECC bounds and proven rigidity thresholds.
\end{remark}

\vfill
\noindent\textbf{Version:} C4--v1.0\\
\textbf{Author:} Inacio F. Vasquez

\end{document}
