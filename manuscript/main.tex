\documentclass[11pt]{article}

\usepackage{amsmath,amssymb,amsthm}
\usepackage{geometry}
\geometry{margin=1in}

\title{Exchange Force Framework}
\author{Inacio F. Vasquez}
\date{}

\begin{document}
\maketitle

\begin{abstract}
We introduce the Exchange Force Framework, a structural theory describing
conservation, capacity, and locality constraints governing exchange processes.
The framework unifies resource, information, and interaction exchanges under
a single invariant formalism.
\end{abstract}

\section{Introduction}
Motivation and scope.

\section{Definitions}
Formal definition of exchange force, admissible systems, and closed exchange.

\section{Core Invariants}
Zero-sum exchange, capacity bounds, locality constraints.

\section{Examples}
Minimal, market, and information exchange examples.

\section{Counterexamples}
Non-operational constructions and boundary failures.

\section{Relation to URF}
Embedding within the Unified Rigidity Framework.

\section{Conclusion}
Implications and future directions.

\bibliographystyle{plain}
\bibliography{references}

\end{document}
