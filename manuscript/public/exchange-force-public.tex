\documentclass[11pt]{article}

\usepackage[a4paper,margin=1in]{geometry}
\usepackage{amsmath,amssymb,amsthm}
\usepackage{hyperref}
\usepackage{enumitem}
\usepackage{microtype}
\usepackage{cite}

\setlist{nosep}

\newtheorem{theorem}{Theorem}
\newtheorem{lemma}{Lemma}
\newtheorem{definition}{Definition}
\newtheorem{proposition}{Proposition}
\newtheorem{corollary}{Corollary}

\title{\bf The Exchange Force:\\
Non--Fundamentality and Rigidity under Finite Capacity}

\author{Inacio Vasquez}
\date{\today}

\begin{document}
\maketitle

\begin{abstract}
We show that exchange forces are not fundamental interactions but emergent,
capacity--limited effects arising from symmetry and mediator elimination.
Building on this classification, we prove that exchange interactions exhibit
rank collapse and correlation rigidity under locality and finite operational
capacity. Exchange forces therefore admit no independent amplification and
occupy a derivative role in physical theory.
\end{abstract}

\section{Introduction}
Motivation, scope, and summary of results.

\section{Exchange Forces Are Not Fundamental}
(Phase I content: definitions, classification, effective Hamiltonian)

\section{Exchange Correlation Capacity}
(ECC definition and assumptions)

\section{Rigidity and Rank Collapse}
(Phase II theorems and consequences)

\section{Physical Consequences}
(condensed matter, nuclear, information-theoretic)

\section{Falsifiability}
(explicit criteria)

\section{Conclusion}

\bibliographystyle{unsrt}
\bibliography{references}

\end{document}
