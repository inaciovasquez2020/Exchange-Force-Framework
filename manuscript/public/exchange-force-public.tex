\documentclass[11pt]{article}

% =========================
% Packages
% =========================
\usepackage[a4paper,margin=1in]{geometry}
\usepackage{amsmath,amssymb,amsthm}
\usepackage{hyperref}
\usepackage{enumitem}
\usepackage{microtype}
\usepackage{cite}

\setlist{nosep}

% =========================
% Theorem Environments
% =========================
\newtheorem{theorem}{Theorem}
\newtheorem{lemma}{Lemma}
\newtheorem{definition}{Definition}
\newtheorem{proposition}{Proposition}
\newtheorem{corollary}{Corollary}

% =========================
% Title
% =========================
\title{\bf The Exchange Force:\\
Non--Fundamentality and Rigidity under Finite Capacity}

\author{Inacio Vasquez}
\date{\today}

\begin{document}
\maketitle

% =========================
% Abstract
% =========================
\begin{abstract}
We show that exchange forces are not fundamental interactions but emergent,
capacity--limited effects arising from symmetry constraints and mediator
elimination. Building on this classification, we prove that exchange
interactions exhibit rank collapse and correlation rigidity once locality,
symmetry, and finite operational capacity are enforced. Exchange forces admit
no independent amplification and occupy a derivative role in physical theory.
\end{abstract}

% =========================
\section{Introduction}
% =========================

Exchange forces appear ubiquitously in quantum physics, from the Heisenberg
interaction in condensed matter systems to effective interactions in nuclear
and atomic models. Despite their prevalence, exchange forces are often treated
as quasi--fundamental dynamical agents.

In this work we show that this interpretation is structurally incorrect.
Exchange forces arise as effective interactions generated by symmetry
constraints or mediator elimination and are therefore non--fundamental.
Moreover, once admissibility constraints are imposed, exchange correlations
exhibit rigidity: their effective dimensionality collapses and admits no
independent proliferation.

The paper proceeds in two phases. Phase~I establishes the non--fundamentality
of exchange forces. Phase~II proves rigidity and rank collapse under finite
Exchange Correlation Capacity (ECC).

% =========================
\section{Exchange Forces Are Not Fundamental}
% =========================

\begin{definition}[Exchange Interaction]
An exchange interaction is an effective interaction term arising from either
(i) symmetry constraints on admissible state spaces or (ii) integration of
mediator degrees of freedom, without introducing a new fundamental coupling.
\end{definition}

\begin{definition}[Operational Setting]
Systems are probed and evolved using admissible operations with bounded
information and energy capacity.
\end{definition}

\begin{theorem}[Non--Fundamentality of Exchange Forces]
No exchange interaction constitutes a fundamental force. All exchange effects
are derivable from symmetry constraints or mediator exchange within an
effective Hamiltonian description.
\end{theorem}

\begin{proof}
Symmetry constraints modify admissible state spaces without introducing new
dynamical degrees of freedom. Mediator elimination produces effective
interactions by integrating out existing fields. Neither mechanism introduces
a new fundamental interaction.
\end{proof}

% =========================
\section{Exchange Correlation Capacity}
% =========================

\begin{definition}[Exchange Correlation Capacity]
The Exchange Correlation Capacity (ECC) is the maximal number of independent
correlation channels realizable via exchange mechanisms under fixed symmetry,
locality, and operational constraints.
\end{definition}

ECC bounds the effective dimensionality of exchange--induced interactions.

\begin{lemma}[Finite ECC]
For any admissible exchange interaction, ECC is finite and independent of
system size.
\end{lemma}

\begin{proof}
Exchange correlations are generated by finite symmetry representations or
finite mediator sectors, each of which yields only finitely many independent
correlation channels under admissible operations.
\end{proof}

% =========================
\section{Rigidity and Rank Collapse}
% =========================

\begin{definition}[Exchange Rank]
The exchange rank is the rank of the effective interaction kernel generated by
exchange correlations.
\end{definition}

\begin{theorem}[Exchange Rank Collapse]
Under locality, symmetry, and finite ECC, the exchange rank collapses to a
finite constant independent of system size.
\end{theorem}

\begin{proof}
Finite ECC bounds the number of independent correlation channels. Each channel
contributes at most one independent interaction mode, forcing finite rank.
\end{proof}

\begin{proposition}[Correlation Rigidity]
Exchange correlations admit no continuous deformation without violating
admissibility constraints.
\end{proposition}

\begin{proof}
Any deformation increasing correlation dimensionality would exceed ECC or
violate locality or symmetry constraints.
\end{proof}

% =========================
\section{Physical Consequences}
% =========================

Exchange rigidity explains the saturation of exchange couplings in condensed
matter systems, the absence of independent exchange forces in nuclear physics,
and bounds on entanglement generation in quantum information.

\begin{corollary}[No Independent Amplification]
Exchange forces cannot be independently amplified without introducing new
mediators or symmetry breaking.
\end{corollary}

% =========================
\section{Falsifiability}
% =========================

The framework would be falsified by an exchange interaction that:
\begin{itemize}
\item Operates under fixed locality and symmetry
\item Exhibits unbounded independent correlation channels
\item Avoids mediator introduction
\end{itemize}

No such interaction is known.

% =========================
\section{Conclusion}
% =========================

Exchange forces are not fundamental interactions but rigid, capacity--limited
effects arising from symmetry and mediator elimination. Their apparent
dynamical freedom is an artifact of effective description. Exchange forces
therefore occupy a constrained, derivative role in physical theory.

% =========================
\appendix
\section{Operator--Algebraic Bounds on Exchange Correlation Capacity}
% =========================

\begin{definition}[Exchange Operator Algebra]
Let $\mathcal{A}_{\mathrm{ex}} \subset \mathcal{B}(\mathcal{H})$ be the unital
$*$--algebra generated by all admissible exchange maps acting on a local
Hilbert space $\mathcal{H}$.
\end{definition}

\begin{definition}[CPTP Exchange Map]
An exchange interaction induces a completely positive trace--preserving map
\[
\Phi(X) = \sum_{i=1}^k K_i X K_i^\dagger
\]
with Kraus operators $\{K_i\}$ determined by symmetry or mediator elimination.
\end{definition}

\begin{lemma}[Sharp ECC Bound]
The Exchange Correlation Capacity satisfies
\[
\mathrm{ECC} \le \dim \mathcal{A}_{\mathrm{ex}} \le k^2,
\]
where $k$ is the maximal Kraus rank over admissible exchange maps.
\end{lemma}

\begin{proof}
All admissible exchange correlations lie in the image of the operator algebra
generated by exchange maps. A CPTP map of Kraus rank $k$ generates an algebra
of dimension at most $k^2$.
\end{proof}

\begin{theorem}[Quantitative Rank Collapse]
For any admissible exchange interaction,
\[
\mathrm{rank}(H_{\mathrm{exchange}}) \le \dim \mathcal{A}_{\mathrm{ex}},
\]
independent of system size.
\end{theorem}

\begin{proof}
Each independent exchange contribution corresponds to an element of
$\mathcal{A}_{\mathrm{ex}}$. Finite algebra dimension enforces rank collapse.
\end{proof}

\bibliographystyle{unsrt}
\bibliography{references}

\end{document}
